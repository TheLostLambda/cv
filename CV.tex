%\nonstopmode
\documentclass[twocolumn, a4paper, fontsize=9pt, headsepline, footsepline]{scrartcl}
\usepackage[
  top=25mm,
  bottom=21mm,
  left=13mm,
  right=13mm,
  headheight=15mm,
  footskip=15.5mm
]{geometry}
\setlength{\footheight}{12mm}

% Change the theme here (light, dark, BW, blue, green)
\usepackage{tagging}
\usetag{##THEME##}
% FIXME: I need to automate generation of the versions here!
% Change what is included in the "EXPERIENCE" section (short, full, bio, comp)
\usetag{short}

\usepackage{fontspec}
\setmainfont[Ligatures=TeX]{Montserrat}
\setmonofont[Scale=0.9]{IBM Plex Mono}

\setlength{\parindent}{5mm}

\usepackage{setspace}
\RedeclareSectionCommands[
  beforeskip=2mm plus 1mm minus 1mm,
  afterskip=1.5mm plus 1mm minus 1mm
]{section,subsection}

\setkomafont{pagehead}{\normalfont}
\setkomafont{section}{\Large\bfseries}
\setkomafont{subsection}{\normalsize}

\usepackage{scrlayer-scrpage}
\usepackage{fontawesome}
\renewcommand*{\faicon}[1]{\makebox[1.5em][c]{\csname faicon@#1\endcsname}}

\ihead{\Huge Brooks J Rady \\\large Student at The University of Sheffield}
\ohead{Sheffield, UK\hspace{2mm}\faMapMarker \\ +44 7413
  390609\hspace{2mm}\faPhoneSquare
  \\ b.j.rady@gmail.com\hspace{2mm}\faEnvelopeSquare} \cfoot{}

% Dr. Stéphane Mesnage, Personal Tutor \& Lecturer in Microbiology <s.mesnage@sheffield.ac.uk>
\ifoot{\textbf{REFERENCES}\\\footnotesize Dr. Egbert Hoiczyk, Project Supervisor
  \& Lecturer in Microbiology <e.hoiczyk@sheffield.ac.uk>\\ Dr. Tuck Seng Wong,
  Lecturer in Chemical \& Biological Engineering <t.wong@sheffield.ac.uk>}
\ofoot{linkedin.com/in/bjrady/\hspace{2mm}\faLinkedinSquare\\thelostlambda.xyz\hspace{2mm}\faGlobe}

\ModifyLayer[addvoffset=-1mm]{scrheadings.foot.above.line}
\ModifyLayer[addvoffset=-1mm]{plain.scrheadings.foot.above.line}
\ModifyLayer[addvoffset=1mm]{scrheadings.head.below.line}
\ModifyLayer[addvoffset=1mm]{plain.scrheadings.head.below.line}

\usepackage{xcolor}
\definecolor{bg}{RGB}{40,40,40}
\definecolor{fg}{RGB}{200,200,200}
\definecolor{rot}{RGB}{244,96,54}
\definecolor{lila}{RGB}{181,68,110}
\definecolor{hellblau}{RGB}{10,205,187}
\definecolor{grun}{RGB}{112,174,110}
\definecolor{link}{RGB}{40,80,180}
\tagged{BW}{\definecolor{rot}{RGB}{0,0,0}}
\tagged{BW}{\definecolor{lila}{RGB}{0,0,0}}
\tagged{BW}{\definecolor{hellblau}{RGB}{0,0,0}}
\tagged{BW}{\definecolor{grun}{RGB}{0,0,0}}
\tagged{BW}{\definecolor{link}{RGB}{0,0,0}}
\tagged{green}{\definecolor{rot}{RGB}{122, 199, 76}}
\tagged{green}{\definecolor{lila}{RGB}{122, 199, 76}}
\tagged{green}{\definecolor{hellblau}{RGB}{122, 199, 76}}
\tagged{green}{\definecolor{grun}{RGB}{122, 199, 76}}
\tagged{blue}{\definecolor{rot}{RGB}{99, 121, 134}}
\tagged{blue}{\definecolor{lila}{RGB}{99, 121, 134}}
\tagged{blue}{\definecolor{hellblau}{RGB}{99, 121, 134}}
\tagged{blue}{\definecolor{grun}{RGB}{99, 121, 134}}
\tagged{dark}{\definecolor{link}{RGB}{110,150,250}}
\tagged{dark}{\pagecolor{bg}}
\tagged{dark}{\color{fg}}

\usepackage[hidelinks, colorlinks=true, urlcolor=link]{hyperref}

\usepackage{soul}

\usepackage{enumitem}
\setlist{nosep}

\begin{document}
\onehalfspacing
\setulcolor{rot}
\setul{}{2pt}
\section*{\ul{OBJECTIVE}}
\noindent
A perpetually curious student with a keen interest in biological research and
computer science. Seeking to engage in research and work alongside professionals
in the field to further mankind's understanding of the world.

\setulcolor{lila}
\section*{\ul{FORMAL EDUCATION}}
\subsection*{Molecular Biology at The University of Sheffield (2019-2022)}
\vspace{-5pt}
\emph{Weighted Average: 86.50\%}\par
\emph{Valerie Broomhead Prize – \emph{Best Level 1 Practical Mark}}\par
\emph{Paul Hancock Prize – \emph{Best Level 2 Practical Mark}}\par

\subsection*{Bioengineering at The University of Sheffield (2018-2019)}
\vspace{-5pt}
\emph{Weighted Average: 89.25\%}\par
\emph{Sir Harold West Award  – \emph{Academic \& Personal Promise}}

\subsection*{Prospect Ridge Academy High School, USA (2014-2018)}
\vspace{-5pt}
\emph{GPA: 4.74 (4.00)}\par
\emph{Graduated Summa Cum Laude}\par
\emph{National AP Scholar Award}\par
\emph{First Place for Senior Capstone Project}

\setulcolor{grun}
\section*{\ul{EXPERIENCE}}

\subsection*{Peptidoglycomics Tool Development (Summer 2021)\\\vspace{-3pt}\textmd{\emph{Placement Work
    in the Mesnage Lab, Sheffield UK}}}
\noindent
A wealth of tools are available for the analysis of mass spectrometry data from
proteomics experiments; however, only one is dedicated to peptidoglycan
structural analysis. I helped improve the recently published PGFinder tool by
integrating it with a downstream MS/MS analysis program developed in the Mesnage
lab.
\begin{itemize}
\item Characterised peptidoglycan via LC-MS and MS/MS
\item Developed a tool for predicting PG fragments
\item Identified and fixed a bug in MS/MS ion generation
\item Validated pipeline via manual spectrum inspection
\end{itemize}

\subsection*{Study of \emph{Myxococcus} Motility \& Sporulation (Autumn 2021)\\\vspace{-3pt}\textmd{\emph{Placement Work
    in the Hoiczyk Lab, Sheffield UK}}}
\noindent
\emph{Myxococcus xanthus} is a model organism for motility, development, and
multicellularity in bacteria. During my project, I generated deletion mutants to
improve the purity of a protein isolation and designed a variety of novel
screens for the identification of a diffusible pheromone involved in fruiting
body formation.
\begin{itemize}
\item Learned plasmid design, cloning, and sequencing
\item Generated deletion mutants via allele exchange
\item Purified protein complexes from whole-cell lysate
\item Verified isolations via SDS-PAGE and Western blot
\item Analysed single molecule assemblies using EM
\item Applied 3D printing to create novel agar moulds
\item Imaged fruiting bodies via light \& stereo microscopy
\end{itemize}
\pagebreak

\subsection*{Peptidoglycan Hydrolase \& \emph{Rhizobium} Reviews
  (2020/2021)\\\vspace{-3pt}\textmd{\emph{Mini-Reviews in Molecular Biology} – \url{https://bit.ly/3qgB6OQ}}}
\noindent
I completed two mini-reviews on the \emph{Rhizobium}-legume symbiosis and the
regulation of peptidoglycan hydrolases. Both papers were limited to 2000 words
of body text, so writing in a clear and concise fashion was critical to
capturing the full picture.
\begin{itemize}
\item Distilled complex topics into digestible sections
\item Managed 70+ in-text references using Zotero
\item Generated figures using PyMOL, GIMP \& Inkscape
\item Professionally typeset the reports in \LaTeX
\end{itemize}

\subsection*{Plant Growth Protocol Standardisation
  (2020-Current)\\\vspace{-3pt}\textmd{\emph{Designer and Developer at Grobotic Systems, Sheffield UK}}}
\noindent
Grobotic Systems is one of many companies producing growth chambers for use in
plant science; unfortunately, there is no standard for how the protocols run by
these chambers should be written. I developed a human and machine readable
protocol format that allows the same protocol to be run by several different
growth chambers.
\begin{itemize}
\item Gained a familiarity with plant growth protocols
\item Worked with other scientists to develop a standard
\item Built a protocol-to-setpoint compiler in Rust
\item Created a web tool for protocol compilation in WASM
\item Added a dynamic JS graph for protocol visualisation
\end{itemize}

\subsection*{Mechanical, Electrical, and Software Engineer
  (2020-Current)\\\vspace{-3pt}\textmd{\emph{Bioreactor Development for Evolutor, Sheffield UK}}}
\noindent
Advised by Dr. Tuck Seng Wong from Chemical and Biological Engineering, I
developed a compact, inexpensive prototype bioreactor optimised for carrying out
directed / adaptive evolution in an automated fashion – allowing for the
optimisation of industrially important microbe strains.
\begin{itemize}
\item Leveraged CAD to prototype a reactor from scratch
\item Manufactured reactor components via 3D printing
\item Developed custom sensor and control electronics
\item Created a dashboard in Svelte for reactor monitoring
\item Experimentally characterised device performance
\item Integrated user feedback into the device design
\end{itemize}

\untagged{short}{
\subsection*{Development Officer \& Workshop Instructor
  (2020-2021)\\\vspace{-3pt}\textmd{\emph{Part of the Sheffield Bionics Society} – \url{https://bit.ly/3qs5xSf}}}
\noindent
Many passionate students join Sheffield Bionics every year but lack the
technical background needed to keep up with the senior members. To address this,
I founded Bionics Bootcamp – designed to get members up to speed with
programming, electronics, and mechanical design.
\begin{itemize}
\item Developed lessons on Python, CAD, ML, and circuitry
\item Created worksheets for reinforcing taught concepts
\item Recorded 30+ hours of live-streamed lesson content
\item Archived the course on a dedicated student website
\end{itemize}
}

\untagged{short}{
\subsection*{Programming Mentor \& Workshop Instructor
  (2018-2021)\\\vspace{-3pt}\textmd{\emph{Volunteer Teaching at the University, Sheffield UK}}}
\noindent
For the past couple of years I've been either helping to teach Python via
organisations like Cookies and Code, or have been running programming workshops
for the Sheffield Bionics Society. I've been both an entry-level and advanced
instructor, working with both individuals and large groups.
\begin{itemize}
\item Communicated complex concepts understandably
\item Demonstrated organisational and interpersonal skills
\item Developed existing skills in computer programming
\item Helped others leverage computation in their research
\end{itemize}}

\subsection*{Software Lead \& Co-Presenter (2019)\\\vspace{-3pt}\textmd{\emph{Member of the
    University iGEM Team, Sheffield UK}}}
\noindent
iGEM is a synthetic biology competition involving universities from around the
globe. Our team designed a low-cost, open-source microplate reader for use in
community labs. I developed the software that powered our device and presented
our project at conferences in Newcastle, UK and Boston, USA.
\begin{itemize}
\item Developed low-level firmware for an ESP32 in C
\item Built a web-interface using HTML, CSS, and JS
\item Displayed an aptitude for project presentation
\item Exercised team communication and planning skills
\item Awarded Best Presentation in the UK
\item Won Best Open Project and Gold Medal in Boston
\end{itemize}

\untagged{short}{
\subsection*{Lost in Translation: Proteins Post-Expression (2019)\\\vspace{-3pt}\textmd{\emph{Scientific Poster} – \url{http://bit.ly/36bKMPG}}}
\noindent
Scanning the literature, I stockpiled information regarding the
post-translational modification and trafficking of proteins. From this body of
research, I picked out the most important points to be presented as part of the
poster — striking the balance between rigour and brevity.
\begin{itemize}
\item Selected key points from a large body of research
\item Demonstrated a competency in graphical design
\item Elucidated the connections between concepts 
\end{itemize}}

\untagged{short}{
\subsection*{Creepy Phenomena: An Investigation of Viscoelasticity
  (2019)\\\vspace{-3pt}\textmd{\emph{Scientific Report} – \url{http://bit.ly/2PkNP2b}}}
\noindent
As part of a university physics course, I wrote a report analysing an
experimental video showcasing a particular viscoelastic model. Working backwards
from only the video and the volume of the syringe, the fundamental constants of
the system were calculated.\par
\begin{itemize}
\item Used maths and statistics to model a physical system
\item Demonstrated writing and problem solving skills
\item Created a report using LaTeX, R, and Knitr
\end{itemize}}

\untagged{short}{
\subsection*{Founder of PRA Robotics (2014-2018)\\\vspace{-3pt}\textmd{\emph{Prospect Ridge
    Academy, Colorado USA}}}
\noindent
The PRA robotics team competed in both the First Tech Challenge (FTC) and the
Autonomous Vehicle Competition (AVC). The team has now grown to several dozen
members and continues foster students' interest in robotics.
\begin{itemize}
\item Led software development for both FTC and AVC
\item Ran Java programming workshops for new recruits
\item Competed as finalists in the FTC State Championship
\item Developed skills in autonomous control engineering
\item Showed leadership skills by managing a large team
\end{itemize}}

\untagged{short}{
\subsection*{\textbf{\texttt{FTC\_HTTP}}: Wireless FTC Robot Programming Tool
  (2017-2021)\\\vspace{-3pt}\textmd{\emph{Cross-platform CLI Application} – \url{http://bit.ly/ftc_http}}}
\noindent
FTC teams used to be forced to choose between two programming toolchains: either
a 5GB+ install and minute-long programming cycles, or an under-featured
web-interface prone to code loss. By reverse engineering the web protocol,
\texttt{FTC\_HTTP} allows wireless development from any text-editor, on any platform, in
less than 8MB.
\begin{itemize}
\item Reverse engineered an undocumented protocol
\item Created a polished, cross-platform application
\item Demonstrated skills in Rust, HTTP, and Git
\item Produced documentation and a video tutorial
\item Shared the application with others during FTC events
\end{itemize}}

\untagged{short}{
\subsection*{Assistant Researcher (2016-2017)\\\vspace{-3pt}\textmd{\emph{Summer Placement at Avidity LLC, Colorado USA}}}
\noindent
During my time at Avidity, I designed a set of DNA tethers for use in
multiplexed pathogen assays, worked on a directed evolution project, and
explored the manufacture of lateral flow assays on an aluminium surface.
\begin{itemize}
\item Developed extensive wet-lab and digital biology skills
\item Applied recombinant DNA technologies
\item Showcased an aptitude for independent research
\end{itemize}}
\pagebreak

\onecolumn
\setulcolor{hellblau}
\section*{\ul{ADDITIONAL WORK}}
\subsection*{Personal Web Server (2015-Current)\\\vspace{-2pt}\hspace{14.22636pt}\textmd{\emph{File-Sync, VPN,
    DLNA, FTP, Games, Blockchain, and Website} —
    \url{http://thelostlambda.xyz/}}}

\tagged{short}{
\subsection*{Workshop Development for Sheffield Bionics Society
  (2020-2021)\\\vspace{-2pt}\hspace{14.22636pt}\textmd{\emph{Video and Blog Content for Teaching Python, CAD, ML, and Electronics} – \url{https://bit.ly/3qs5xSf}}}}

\subsection*{Haskell Fives-and-Threes Dominoes AI (2020)\\\vspace{-2pt}\hspace{14.22636pt}\textmd{\emph{A
    Tournament-Winning Dominoes AI and Accompanying Performance Report} —
    \url{https://bit.ly/3qoQpVx}}}

\subsection*{Sickle Cell Disease: Symptoms, Molecular Basis, and Treatments
(2019)\\\vspace{-2pt}\hspace{14.22636pt}\textmd{\emph{A Mini-Review Focused on Sickle Cell
    Pathogenesis} — \url{http://bit.ly/2E7Yu9x}}}

\tagged{short}{
\subsection*{Lost in Translation: Proteins Post-Expression
  (2019)\\\vspace{-2pt}\hspace{14.22636pt}\textmd{\emph{Scientific Poster on the
    Post-Translational Modification and Transport of Proteins} —
    \url{http://bit.ly/36bKMPG}}}}

\subsection*{Under Pressure — Hydrostatics and Elastic Tube Distention
  (2019)\\\vspace{-2pt}\hspace{14.22636pt}\textmd{\emph{Modelling a Biophysical System from
    Image Data} — \url{http://bit.ly/2qMQmYA}}}

\tagged{short}{
\subsection*{Creepy Phenomena: An Investigation of Viscoelasticity
  (2019)\\\vspace{-2pt}\hspace{14.22636pt}\textmd{\emph{Mathematical Analysis of a Physical
    System from Video Footage} — \url{http://bit.ly/2PkNP2b}}}}

\subsection*{Tissue Who? — A Foray into Histology and Tissue Identification
  (2019)\\\vspace{-2pt}\hspace{14.22636pt}\textmd{\emph{A Short Histology Primer \& Lab
    Report} — \url{http://bit.ly/2ooernJ}}}

\subsection*{In Vivo Detection and Signaling of Arbitrary DNA Sequences
  (2018)\\\vspace{-3pt}\hspace{14.22636pt}\textmd{\emph{High School Capstone Project Exploring
    a Novel Use for CRISPR} — \url{http://bit.ly/2pbA9fk}}}

\subsection*{Pokéstats — What Type Of Pokemon Is The Match For You?
  (2018)\\\vspace{-2pt}\hspace{14.22636pt}\textmd{\emph{A Statistical Report Exploring a
    Large Pokémon Dataset} — \url{http://bit.ly/2FVjMqh}}}

\subsection*{The Regicide of the Fisher King (2018)\\\vspace{-2pt}\hspace{14.22636pt}\textmd{\emph{AP English
    Literature Modernism Essay} — \url{http://bit.ly/2FHoYSy}}}

\subsection*{Honors Physics “Build a Planet” Project (2017)\\\vspace{-2pt}\hspace{14.22636pt}\textmd{\emph{Exploring and
    Mathematically Modelling the Physics of a Fictional Planet} —
    \url{http://bit.ly/2IA9f5F}}}

\tagged{short}{
\subsection*{\textbf{\texttt{FTC\_HTTP}} Programming Tool (2017)\\\vspace{-2pt}\hspace{14.22636pt}\textmd{\emph{A Cross-Platform
    Application for the Wireless Programming of FTC Robots} —
    \url{http://bit.ly/ftc_http}}}}

\setulcolor{rot}
\section*{\ul{AWARDS \& CERTIFICATES}}
\begin{itemize}
\item Phil Green Trophy for Best Haskell Dominoes AI in COM2108 Tournament (2020)
\item Won Best Open Project \& Gold Medal during iGEM Jamboree (2019)
\item Awarded Best Presentation during iGEM UK Meetup (2019)
\item Best Communicated Solution during EWB Global Engineering Challenge (2019)
\item Linnaeus Award for Excellence in Biology (2018)
\item Hacker Award for Excellence in Computer Science (2018)
\item National Honor Society (2017-2018)
\item Prospect Ridge Academy High Honor Roll (2015-2018)
\item Design Award \& Finalist Alliance at FTC State Championship (2017-2018)
\item Second place in Junior Energy and Transportation at Colorado Science and
  Engineering Fair (2014)
\item First place in Alternative Fuels at Denver Metro Science Fair (2014)
\end{itemize}
\setulcolor{grun}
\section*{\ul{HOBBIES \& INTERESTS}}
\noindent
I'm into anything that bytes – I built my first PC at 14 and have used it to
teach myself programming, learn how to manage a Linux server, and enjoy some
video games. Though I don't care much for commercial projects, I've spent a
large amount of my time contributing to open-source: developing infrastructure
and course content for Exercism, authoring large portions of the Zellij terminal
application, and countless smaller contributions.

Alongside this has been an enduring interest in robotics: I founded PRA robotics
at my high school and have been part of Avalon, Bionics, and Project Hex since
coming to Uni – three different student-led robotics groups. My passion for
learning extends to teaching, and I've represented my course as academic
officer, tutored first year biologists, and delivered lessons in coding,
engineering, and ethical hacking all across campus.

I enjoy both hiking and rock climbing, and though I can't draw to save my life,
I've recently taken up digital photography and enjoy graphic design. I've been
learning German for nearly four years now and my 125,902 minutes of listening to
music in 2020 means I'm probably a bit obsessed with ``Midwestern Emo'' –
unfortunately, I'm still waiting for the philosophy I've been reading and
writing about to make sense of all that...
\pagebreak
\end{document}
 
% DIRTY CHUNK !!!
\tagged{comment}{
% Eventually I need to add a section for my L3 project – unfortunately I'm not
% quite at the point where I can link to anything there...

New Experience:

% Add stuff about Bionics here (taught programming, electronics, and design)
% https://bit.ly/3esFyUZ

% Evolutor is my engineering magnum opus so far, 3D design, electronics, and software

% PGFinder work + Graph Generator + MS/MS pipeline with Mesnage

% Growbotics work is an in to the plant side of things!

% Add my essays about Rhizobia & Hydrolases in one section here
% https://bit.ly/3qgB6OQ

% Keep iGEM!

-- LESS IMPORTANT STUFF --

% Bionics ML work with Thomas Thomas

% Ethical Hacking, more teaching experience in Cyber security / Linux fields

% Exercism is a nice bit of open source teaching and programming work

% Avalon was some electrical engineering experience and practice making PCBs

% Zellij is my primary open-source project now – lots of good, international teamwork

% MBB society as academic officer (working with staff)

% PAL teaching of biology to first-year students

% Additional Work

% Move the poster and report from bioengineering + ftc\_http here (keep long
% version too behind a generation flag!)

% Zellij and Exercism will likely end up here!

% Add my Haskell Domino project! https://bit.ly/3qoQpVx

Miscellaneous

Add my nationality somewhere (so people know I'm international status)

Add languages as well (mostly just English, obviously, but with a bit of German)

% Other open-source stuff could be worth mentioning...

Starting up iGEM again is something that's noteworthy!

% Project Hex, more engineering / robotics / programming

% Add rock climbing to hobbies and interests!

Hobbies:

%Computers (Building, Games, Programming, Homelab)
%Photography
%Hiking
%Reading / Writing (Technical, Political / Philosophical, Fiction)
%Rock climbing / bouldering
Music (indie rubbish for years)
%German!
}
% END DIRTY CHUNK
